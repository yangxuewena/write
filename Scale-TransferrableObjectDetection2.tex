\documentclass[10pt,twocolumn,letterpaper]{article}

\usepackage{cvpr}
\usepackage{times}
\usepackage{epsfig}
\usepackage{graphicx}
\usepackage{amsmath}
\usepackage{amssymb}

\usepackage{times}


\cvprfinalcopy % *** Uncomment this line for the final submission
\def\cvprPaperID{****} % *** Enter the CVPR Paper ID here
\def\httilde{\mbox{\tt\raisebox{-.5ex}{\symbol{126}}}}

\usepackage{indentfirst}
\setlength{\parindent}{2em}
\usepackage{cite}
\usepackage[colorlinks,linkcolor=red,anchorcolor=blue,citecolor=green,backref=page]{hyperref}

\author{Xuewen Yang\\\\
August 19 2018}

\title{Scale-Transferrable Object Detection}
\begin{document}
\maketitle

\section{Scale-Transferrable Detection Network}
In this section,the author first introduce the base network which
is our feature extraction network component. They use
DenseNet\cite{Huang2017Densely} as our base network. In each dense block
of DenseNet, for each layer, its feature map is used as inputs
for all subsequent layers. The output of the last layer
of the dense block has highest number of channels and is
suitable as input for our scale-transfer layer which expands
the width and height of the feature map by compressing the
number of channels. Then the author describe the scale-transfer
module that produces feature maps at different scales. Next,they describe the entire object detection/location prediction
network architecture and the network training details.
\subsection{Base Network : DenseNet}
The author use DenseNet-169\cite{Huang2017Densely} as the base network for feature
extraction and do pre-training on the ILSVRC CLSLOC
dataset\cite{Russakovsky2015ImageNet}. DenseNet is a network with deep supervision.
In each dense block of DenseNet, the output of
each layer contains the output of all previous layers, and
thus incorporates low-level and high-level features of the
input image, which is suitable for object detection. Inspired
by DSOD, the author replace the input layers (7$\times$7 convolution
layer, stride = 2 followed by a 3$\times$3 max pooling
layer, stride = 2) into three 3$\times$3 convolution layers and one
2$\times$2 mean pooling layer. The stride of the first convolution
layer is 2 and the others are 1. The output channels for all
three convolution layers are 64. They call these layers ��stem
block��.Experiments show that this simple substitution can significantly
improve the accuracy of object detection. One explanation could be that the input
layers in the original DenseNet-169 have lost much information
due to two consecutive down sampling. This will
impair the performance of object detection, especially for
small objects.
\subsection{High Efficiency ScaleTransfer Module}
Scale problem lies in the heart of object detection. Combining
predictions from multiple feature maps with different
resolutions are beneficial for detecting multi-scale objects.
However, as shown in Figure~\ref{fig:1}, in the last dense block of
DenseNet, all outputs of layers have the same width and
height, except for the number of channels. For example,
when the input image is 300$\times$300, the last dense block dimension
of DenseNet-169 is 9$\times$9. A simple approach is to
predict directly using high-resolution feature maps of low
layers, similar to SSD [22]. However, the low-level feature
map lacks semantic information about objects, which may
cause low performance on object detection.
\begin{figure}
\centering
\includegraphics[width=8cm,height=6cm]{1}
\caption{Importance of context for object recognition. Without the context (face), it is hard to recognize the black curve in the middle area as a nose.}
\label{fig:1}
\end{figure}
In order to obtain different resolution feature maps with
strong semantic information, the author develop a module named scale-transfer module. Scale-transfer module is very efficient
and can be directly embedded into the dense block in
DenseNet. To get strong semantic feature maps, they make
use of the network structure of DenseNet to transfer the lowlevel
features directly to the top of the network through the
concat operation. The feature map at the top of the network
has both low-level detail information and high-level semantic
information so as to improve the performance of object
localization and classification.

They get feature maps of different scales from the last
dense block of DenseNet. In scale-transfer module, they
use the mean pooling layer to obtain low-resolution feature
maps. For high-resolution feature maps, they use a technique
called scale-transfer layer.

{\small
\bibliographystyle{ieee}
\bibliography{1}
}

\end{document}
