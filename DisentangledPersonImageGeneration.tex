\documentclass[10pt,twocolumn,letterpaper]{article}

\usepackage{cvpr}
\usepackage{times}
\usepackage{epsfig}
\usepackage{graphicx}
\usepackage{amsmath}
\usepackage{amssymb}

\usepackage{times}


\cvprfinalcopy % *** Uncomment this line for the final submission
\def\cvprPaperID{****} % *** Enter the CVPR Paper ID here
\def\httilde{\mbox{\tt\raisebox{-.5ex}{\symbol{126}}}}

\usepackage{indentfirst}
\setlength{\parindent}{2em}
\usepackage{cite}
\usepackage[colorlinks,linkcolor=red,anchorcolor=blue,citecolor=green,backref=page]{hyperref}

\author{Xuewen Yang\\\\
July 22 2018}

\title{Disentangled Person Image Generation}
\begin{document}
\maketitle
\begin{abstract}
Generating novel, yet realistic, images of persons is a
challenging task due to the complex interplay between the
different image factors, such as the foreground, background
and pose information. In this work,the author aim at generat-
ing such images based on a novel, two-stage reconstruc-
tion pipeline that learns a disentangled representation of
the aforementioned image factors and generates novel per-
son images at the same time.First, a multi-branched recon-
struction network is proposed to disentangle and encode the
three factors into embedding features, which are then com-
bined to re-compose the input image itself. Second, three
corresponding mapping functions are learned in an adver-
sarial manner in order to map Gaussian noise to the learned
embedding feature space, for each factor, respectively.
\end{abstract}
\section{Introduction}
The author disentangle the input image into intermediate
embedding features, i.e. person images can be reduced
to a composition of features of foreground, background, and
pose. Compared to existing approaches,they rely on a different
technique to generate new samples. In particular,they
aim at sampling from a standard distribution, e.g. a Gaussian
distribution, to first generate new embedding features
and from them generate new images.
\section{Method}
The goal is to disentangle the appearance and structure
factors in person images, so that they can manipulate the foreground,
background and pose separately. To achieve this,
the author propose a two-stage pipeline shown in Figure~\ref{fig:1}. In stage-I,
they disentangle the foreground, background and pose factors
using a reconstruction network in a divide-and-conquer
manner. In particular,they reconstruct person images by
first disentangling into intermediate embedding features of the three factors, then recover the input image by decoding
these features. In stage-II,they treat these features as real to
learn mapping functions for mapping a Gaussian distribution
to the embedding feature distribution adversarially.

\subsection{StageI:Disentangled image reconstruction}
At stage-I, they propose a multi-branched reconstruction
architecture to disentangle the foreground, background and
pose factors.

To separate the foreground and background
information, they apply the coarse pose mask to the
feature maps instead of the input image directly. By doing
so, they can alleviate the inaccuracies of the coarse pose
mask. Then, in order to further disentangle the foreground
from the pose information, they encode pose invariant features
with 7 Body Regions-Of-Interest instead of the whole
image similar to\cite{Zhao2017Spindle}.

For the background branch,the author apply
the inverse pose mask to get the background feature
maps and pass them into the background encoder to obtain
a 128-dim embedding feature. Then, the foreground
and background features are concatenated and tiled into
$128\times64\times352$ appearance feature maps.

For the pose branch,they concatenate the 18-
channel heatmaps with the appearance feature maps and
pass them into the a ��U-Net��-based architecture\cite{ronneberger2015u-net:}, i.e.,
convolutional autoencoder with skip connections, to generate
the final person image following $PG^2 (G1+D)$\cite{ma2017pose}.
\begin{figure}
\centering
\includegraphics[width=8cm,height=6cm]{1}
\caption{The two-stage framework.}
\label{fig:1}
\end{figure}
\subsection{StageII:Embedding feature mapping}
They propose a two-step mapping
technique as illustrated in Figure~\ref{fig:1}. Instead of directly learning
to decode Gaussian noise to the image space, they first
learn a mapping function $\Phi$ that maps a Gaussian space
Z into a continuous feature embedding space E, and then
use the pre-trained decoder to map the feature embedding
space E into the real image space X. The encoder learned
in stage-I encodes the FG, BG and Pose factors x into low-dimensional
real embedding features e. Then, they treat the
features mapped from Gaussian noise z as fake embedding
features and learn the mapping function $\Phi$ adversarially.
In this way, we can sample fake embedding features from
noise and then map them back to images using the decoder
learned in stage-I. The proposed two-step mapping technique
is easy to train in a piecewise style and most importantly
can be useful for other image generation applications.

{\small
\bibliographystyle{ieee}
\bibliography{1}
}

\end{document}
