\documentclass[10pt,twocolumn,letterpaper]{article}

\usepackage{cvpr}
\usepackage{times}
\usepackage{epsfig}
\usepackage{graphicx}
\usepackage{amsmath}
\usepackage{amssymb}

\usepackage{times}


\cvprfinalcopy % *** Uncomment this line for the final submission
\def\cvprPaperID{****} % *** Enter the CVPR Paper ID here
\def\httilde{\mbox{\tt\raisebox{-.5ex}{\symbol{126}}}}

\usepackage{indentfirst}
\setlength{\parindent}{2em}
\usepackage{cite}
\usepackage[colorlinks,linkcolor=red,anchorcolor=blue,citecolor=green,backref=page]{hyperref}

\author{Xuewen Yang\\\\
August 7 2018}

\title{Shape Driven Kernel Adaptation in Convolutional Neural Network for Robust Facial Trait Recognition}
\begin{document}
\maketitle
\begin{abstract}
One key challenge of facial trait recognition is the large
non-rigid appearance variations due to some irrelevant real
world factors, such as viewpoint and expression changes.
In this paper, the author explore how the shape information, i.e. facial
landmark positions, can be explicitly deployed into the
popular Convolutional Neural Network (CNN) architecture
to disentangle such irrelevant non-rigid appearance variations.
First, instead of using fixed kernels, they propose a kernel
adaptation method to dynamically determine the convolutional
kernels according to the spatial distribution of facial
landmarks, which helps learning more robust features.
Second, motivated by the intuition that different local facial
regions may demand different adaptation functions, they
further propose a tree-structured convolutional architecture
to hierarchically fuse multiple local adaptive CNN subnetworks.
Comprehensive experiments on WebFace, Morph II
and MultiPIE databases well validate the effectiveness of
the proposed kernel adaptation method and tree-structured
convolutional architecture for facial trait recognition tasks,
including identity, age and gender recognition. For all the
tasks, the proposed architecture consistently achieves the
state-of-the-art performances.
\end{abstract}
\section{Introduction}
In the last decade, great progress has been made in developing
deep neural network for various computer vision
tasks. Among them, Convolutional Neural Network
(CNN) has achieved exciting performance on
digit recognition\cite{Schmidhuber2012Multi}, traffic sign recognition\cite{Schmidhuber2012Multi}, object
recognition\cite{Krizhevsky2012ImageNet} and scene labeling.

In this paper, the author also propose to exploit the potential
of facial shape information, i.e. a set of facial landmarks,
to help CNN based methods learning more powerful and
robust face representation. The basic idea is using different
convolutional kernel according to the shape information,
i.e. distribution of face landmarks, in order that the learned
features would become more invariant to appearance variations
caused by different viewpoints or expressions. Another
potential of this kernel adaptation is that the additional
discriminant information contained in the shape may also
be directly coded into the learned feature. Specifically, they
propose a shape driven kernel adaptation for CNN and use
automatically adapted kernels to more efficiently disentangle the mixed factors in each input face image.
\section{Kernel Adaptation for CNN}
In real world environment, facial appearance may
change significantly due to different poses and expressions,
one traditional convolutional layer with fixed kernel functions
may generate undesirably different responses for the
same face. To achieve feature invariance under these complex
variations, a mechanism, which can make the convolution
kernel automatically adapt to the specific variations of
each face instance, will be beneficial.

To this end, they propose a kernel adaptation mechanism
for traditional CNN framework. Suppose the input face
image is I and the kernel function is f, they hope the kernel
function can be automatically adapted for the input face image
according to a latent variable S. Then the convolution
with kernel adaptation can be formulated as:
\begin{equation}
f=\phi(S,\theta)
\end{equation}
\begin{equation}
C=\varphi(I*f+b)
\end{equation}
\section{Conclusion}
In this paper, the author propose a kernel adaptation method in
CNN to exploit shape information for disentangling irrelevant
non-rigid facial appearance variations. Since different
facial regions have different deformations, to better exert its
function, they adopt kernel adaptation in multiple local regions
respectively and further propose a tree-structured convolutional
architecture to jointly learn features in an end-toend
manner. Evaluations on facial trait recognition tasks
demonstrate the state-of-the-art performances of the proposed
tree-a-CNN model.

Although this network has relatively shallow structure
comparing to the state-of-the-art deep convolutional neural
networks\cite{Krizhevsky2012ImageNet}, our method achieves comparable or
better performance than AlexNet\cite{Krizhevsky2012ImageNet} and DeepFace\cite{Taigman2014DeepFace}.
These results suggest that kernel adaptation method provides
a more compact and effective way to disentangle complex
factors in facial images. This is very helpful for an
large end-to-end system like deep networks. In future, they
will try to deploy this kernel adaptation method into larger
and deeper networks to fully explore the potential of shape
information for robust feature learning.
{\small
\bibliographystyle{ieee}
\bibliography{1}
}

\end{document}
