\documentclass[10pt,twocolumn,letterpaper]{article}

\usepackage{cvpr}
\usepackage{times}
\usepackage{epsfig}
\usepackage{graphicx}
\usepackage{amsmath}
\usepackage{amssymb}

\usepackage{times}


\cvprfinalcopy % *** Uncomment this line for the final submission
\def\cvprPaperID{****} % *** Enter the CVPR Paper ID here
\def\httilde{\mbox{\tt\raisebox{-.5ex}{\symbol{126}}}}

\usepackage{indentfirst}
\setlength{\parindent}{2em}
\usepackage{cite}
\usepackage[colorlinks,linkcolor=red,anchorcolor=blue,citecolor=green,backref=page]{hyperref}

\author{Xuewen Yang\\\\
August 5 2018}

\title{A Dynamic Programming Approach for Fast and Robust Object Pose Recognition from Range Images}
\begin{document}
\maketitle
\begin{abstract}
Joint object recognition and pose estimation solely from
range images is an important task e.g. in robotics applications
and in automated manufacturing environments. The
lack of color information and limitations of current commodity
depth sensors make this task a challenging computer
vision problem, and a standard random sampling based approach
is prohibitively time-consuming.The author propose to address
this difficult problem by generating promising inlier
sets for pose estimation by early rejection of clear outliers
with the help of local belief propagation (or dynamic programming).
By exploiting data-parallelism this method is
fast, and they also do not rely on a computationally expensive
training phase.They demonstrate state-of-the art performance
on a standard dataset and illustrate our approach on
challenging real sequences.
\end{abstract}
\section{Introduction}
Object detection from 3D inputs has been widely researched
during the past decade. Initially many solutions
focused on trying to solve object detection from
laser scans or even from synthetically generated meshes\cite{Drost2010Model}\cite{Johnson2002Using}. However, with the popularization of RGB-D
sensors since 2010 there has been an increasing demand of
algorithms that operate at interactive frame
rates and that are able to cope with inputs that are less reliable
than laser scans. Most of the latter algorithms rely
heavily on RGB data to perform detection, which prohibits
the application of these methods on 3D only inputs. Several
approaches use a global description of the object
(using RGB edges and depth normals), hence these methods
have difficulties in handling occlusions.Methods that utilize only 3D data as input can be based
on either global or local object representations. Several proposed
methods based on a global object representation employ
the Hough transform. These approaches
create a set of features that are accumulated in a Hough voting
space and then select the pose which gathered the largest
number of votes.Mian~\emph{ et al.}\cite{Mian2006Three}
also use the normal to obtain an invariant descriptor: they
fix the local coordinate frame by using two points on the
model (in additional to the normal), and fill an occupancy
grid given the local coordinate frame.
\section{Approach}
\subsection{Descriptor Computation}
A natural choice for a descriptor to represent (local) geometry
is based on an implicit volumetric representation of
range images and 3D surface meshes.The author employ a binary
occupancy grid to compute descriptors. A slightly
more discriminative volumetric data structure would be a
(truncated) signed distance function (TSDF), but they discard
this option for efficiency reasons (proper TSDF computation
is costly, and the descriptors would use several bits per
voxel).they believe that using generalizations of successful
gradient-based image descriptors to 3D shapes (such as 3D-SURF\cite{Knopp2010Hough}) is not necessary, since the intensity values of
the (3D) image are known to be only 0 and 1 for occupancy
grids (and therefore invariance to intensity transformations
is unnecessary). Consequently, this descriptor is a bit string of occupancies in the vicinity of a surface point.
\subsection{Matching}
At test time descriptors are computed for each pixel
with valid depth and estimated surface normal in the (subsampled)
depth image, and the task is to efficiently determine
the set of object coordinates with similar local shape
appearance. The natural choice to quantify similarity of binary
strings is the Hamming distance. The author experimented
with approximated nearest neighbours implementation for
binary data in FLANN\cite{Muja2012Fast} and with a hashing based indexing
data structure using orthonormal projections.
Since in experience the performance is roughly similar
for both acceleration strategies,they only report the results
using FLANN below.
\section{Conclusions}
The author have addressed the problem of 3D object detection
and corresponding pose estimation, and they discussed a
more efficient paradigm to solve this task while still obtaining
state of the art detection rates.The author believe that this work
creates a new and robust framework from which to build
new 3D object detection approaches. In this current work
they left out basically any learning-based technique to boost
the detection performance or run-time behavior. While they
argue that computationally expensive learning techniques
will limit the general applicability of 3D object recognition
(since adding new objects requires time-consuming retraining),
they foresee that more sophisticated processing of training
objects than this current one will lead to more discriminative
descriptors, and therefore will be highly beneficial
for this task.

{\small
\bibliographystyle{ieee}
\bibliography{1}
}

\end{document}
 