\documentclass[10pt,twocolumn,letterpaper]{article}

\usepackage{cvpr}
\usepackage{times}
\usepackage{epsfig}
\usepackage{graphicx}
\usepackage{amsmath}
\usepackage{amssymb}

\usepackage{times}


\cvprfinalcopy % *** Uncomment this line for the final submission
\def\cvprPaperID{****} % *** Enter the CVPR Paper ID here
\def\httilde{\mbox{\tt\raisebox{-.5ex}{\symbol{126}}}}

\usepackage{indentfirst}
\setlength{\parindent}{2em}
\usepackage{cite}
\usepackage[colorlinks,linkcolor=red,anchorcolor=blue,citecolor=green,backref=page]{hyperref}

\author{Xuewen Yang\\\\
July 4 2018}

\title{On the Appearance of Translucent Edges}
\begin{document}
\maketitle
\begin{abstract}
Edges in images of translucent objects are very different from edges in images of opaque objects. The physical causes for these differences are hard to characterize analytically and are not well understood.This paper considers one class of translucency edges��those caused by a discontinuity in surface orientation��and describes the physical causes of their appearance.The author simulate thousands of translucency edge profiles using many different scattering material parameters,and the author explain the resulting variety of edge patterns by qualitatively analyzing light transport.The author also discuss the existence of shape and material metamers, or combinations of distinct shape or material parameters that generate the same edge profile. This knowledge is relevant to visual inference tasks that involve translucent objects,such as shape or material estimation.
\end{abstract}
\section{Introduction}
Translucency is a common visual phenomenon. It occurs
whenever light penetrates a material and scatters within it
before re-emerging toward the observer. This internal scattering
can create a variety of image effects, depending on an
object��s shape and material; its distance from the observer;
and the composition of the lighting around it. Common human
experience suggests that these image effects contain
useful material information, and there is psychophysical evidence
that humans can discriminate subtle differences in
translucent appearance, recognize translucent material categories,
and make inferences about physical scattering parameters\cite{Fleming2005Low}\cite{Gkioulekas2013Understanding}\cite{Xiao2014Looking}.

One prominent class of brightness patterns is edges, or sharp local changes in image brightness. Edges have a variety of causes��cast shadows, material boundaries, occlusions,etc.��and models of edges play prominent roles in a variety of inference algorithms, including contour detection,
deblurring, and material recognition. In this paper,the author focus on edges that are caused by a discontinuous change in surface orientation, such as at the corners of the cubes in Figure~\ref{fig:1}. Locally, this geometry can be modeled by the onedimensional wedge of Figure~\ref{fig:2}. The author use this as an archetypal configuration for studying edges of translucency.

The radiance profile observed from an opaque wedge would be the familiar step function, which is a popular edge model in image processing and computer vision. But the radiance profiles for translucent wedges are very different. As shown in Figure~\ref{fig:1}, they tend to exhibit multiple extrema in the vicinity of an orientation discontinuity, and these extrema are often displaced away from the geometric discontinuity.
\begin{figure}
\centering
\includegraphics{1}
\caption{Radiance profiles at a discontinuity in orientation.
The wedge configuration is a horizontal cross-section of two
planar faces that meet in a vertical line. Here it is lit from
the left. If the material were opaque (as below) the profile
would be a step edge. When it is translucent (as above), the
profiles can be very different, often with multiple extrema
that are displaced from the discontinuity.}
\label{fig:1}
\end{figure}
\section{Scattering near an orientation discontinuity}
The author consider a wedge with a perfectly sharp corner, made of a single, homogeneous translucent material. A two-dimensional cross-section of this scene is shown in Figure~\ref{fig:2}.The wedge is illuminated from the side by collimated light,and is imaged from above by an orthographic camera. This geometry has three degrees of freedom: the illumination and view directions $��_l$, $��_v$, and the wedge angle $��_w$.
\begin{figure}
\centering
\includegraphics{2}
\caption{ Left: Representative edge radiance profile for
a translucent material. A profile is typically made up of
four qualitatively distinct regions, highlighted with different
background colors. These give rise to characteristic
features, in the form of local radiance extrema away from
the geometric edge, as shown. Right: Decomposition of
the profile into single, mid-order, and high-order scattering
components. Different subsets of the four qualitative regions
of the full profile appear in the three scattering components,
indicated by matching colors.}
\label{fig:2}
\end{figure}
Scattering is governed by three material parameters that appear in the radiative transfer equation: the density (extinction coefficient); the scattering albedo; and the scattering phase function. Density ��t controls the spatial frequency of volume events. Albedo is a probability controlling whether light at a volume event is scattered or absorbed. Finally,phase function p is a probability distribution function on the sphere of directions that describes the angular distribution of light scattered at volume events. As is usual, we assume the phase function is cylindrically-symmetric and invariant to rotations of the incident direction, so it only depends on angle $��$ relative to the incident direction.
{\small
\bibliographystyle{ieee}
\bibliography{1}
}

\end{document}
