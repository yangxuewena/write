\documentclass[10pt,twocolumn,letterpaper]{article}

\usepackage{cvpr}
\usepackage{times}
\usepackage{epsfig}
\usepackage{graphicx}
\usepackage{amsmath}
\usepackage{amssymb}

\usepackage{times}


\cvprfinalcopy % *** Uncomment this line for the final submission
\def\cvprPaperID{****} % *** Enter the CVPR Paper ID here
\def\httilde{\mbox{\tt\raisebox{-.5ex}{\symbol{126}}}}

\usepackage{indentfirst}
\setlength{\parindent}{2em}
\usepackage{cite}
\usepackage[colorlinks,linkcolor=red,anchorcolor=blue,citecolor=green,backref=page]{hyperref}

\author{Xuewen Yang\\\\
July 4 2018}

\title{On the Appearance of Translucent Edges}
\begin{document}
\maketitle
\begin{abstract}
Edges in images of translucent objects are very different from edges in images of opaque objects. The physical causes for these differences are hard to characterize analytically and are not well understood.This paper considers one class of translucency edges��those caused by a discontinuity in surface orientation��and describes the physical causes of their appearance.The author simulate thousands of translucency edge profiles using many different scattering material parameters,and the author explain the resulting variety of edge patterns by qualitatively analyzing light transport.The author also discuss the existence of shape and material metamers, or combinations of distinct shape or material parameters that generate the same edge profile. This knowledge is relevant to visual inference tasks that involve translucent objects,such as shape or material estimation.
\end{abstract}
\section{Introduction}
Translucency is a common visual phenomenon. It occurs
whenever light penetrates a material and scatters within it
before re-emerging toward the observer. This internal scattering
can create a variety of image effects, depending on an
object��s shape and material; its distance from the observer;
and the composition of the lighting around it. Common human
experience suggests that these image effects contain
useful material information, and there is psychophysical evidence
that humans can discriminate subtle differences in
translucent appearance, recognize translucent material categories,
and make inferences about physical scattering parameters
[2, 11, 13, 27].


\section{Scattering near an orientation discontinuity}
We consider a wedge with a perfectly sharp corner, made
of a single, homogeneous translucent material. A twodimensional
cross-section of this scene is shown in Figure 2.
The wedge is illuminated from the side by collimated light,
and is imaged from above by an orthographic camera. This
geometry has three degrees of freedom: the illumination
and view directions ��l
, ��v, and the wedge angle ��w.



\section{Conclusion}

{\small
\bibliographystyle{ieee}
\bibliography{1}
}

\end{document}
