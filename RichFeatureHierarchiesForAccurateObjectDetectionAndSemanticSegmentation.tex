\documentclass[10pt,twocolumn,letterpaper]{article}

\usepackage{cvpr}
\usepackage{times}
\usepackage{epsfig}
\usepackage{graphicx}
\usepackage{amsmath}
\usepackage{amssymb}

\usepackage{times}


\cvprfinalcopy % *** Uncomment this line for the final submission
\def\cvprPaperID{****} % *** Enter the CVPR Paper ID here
\def\httilde{\mbox{\tt\raisebox{-.5ex}{\symbol{126}}}}

\usepackage{indentfirst}
\setlength{\parindent}{2em}
\usepackage{cite}
\usepackage[colorlinks,linkcolor=red,anchorcolor=blue,citecolor=green,backref=page]{hyperref}

\author{Xuewen Yang\\\\
August 1 2018}

\title{Rich feature hierarchies for accurate object detection and semantic segmentation}
\begin{document}
\maketitle
\begin{abstract}
Object detection performance, as measured on the
canonical PASCAL VOC dataset, has plateaued in the last
few years. The best-performing methods are complex ensemble
systems that typically combine multiple low-level
image features with high-level context. In this paper, the author
propose a simple and scalable detection algorithm that improves
mean average precision (mAP) by more than 30\%
relative to the previous best result on VOC 2012��achieving
a mAP of 53.3\%.Since they combine region proposals
with CNNs, they call our method R-CNN: Regions with CNN
features. They also compare R-CNN to OverFeat, a recently
proposed sliding-window detector based on a similar CNN
architecture. They find that R-CNN outperforms OverFeat
by a large margin on the 200-class ILSVRC2013 detection
dataset.
\end{abstract}
\section{Introduction}
CNNs saw heavy use in the 1990s (e.g., [27]), but then
fell out of fashion with the rise of support vector machines.
In 2012, Krizhevsky~\emph{et al.}\cite{Krizhevsky2012ImageNet} rekindled interest in CNNs
by showing substantially higher image classification accuracy
on the ImageNet Large Scale Visual Recognition Challenge
(ILSVRC)\cite{Deng2009ImageNet}. Their success resulted from training
a large CNN on 1.2 million labeled images, together
with a few twists on LeCun��s CNN.

Unlike image classification, detection requires localizing
(likely many) objects within an image. One approach
frames localization as a regression problem. However, work
from Szegedy~\emph{et al.}\cite{Szegedy2013Deep}, concurrent with this method, indicates
that this strategy may not fare well in practice (they
report a mAP of 30.5\% on VOC 2007 compared to the
58.5\% achieved by this method). An alternative is to build a
sliding-window detector. CNNs have been used in this way
for at least two decades, typically on constrained object categories,
such as faces and pedestrians\cite{Sermanet2013Pedestrian}. In order
to maintain high spatial resolution, these CNNs typically
only have two convolutional and pooling layers. The author also
considered adopting a sliding-window approach. However,
units high up in our network, which has five convolutional
layers, have very large receptive fields (195$\times$195 pixels)
and strides (32$\times$32 pixels) in the input image, which makes
precise localization within the sliding-window paradigm an
open technical challenge.
\section{Object detection with R-CNN}
This object detection system consists of three modules.
The first generates category-independent region proposals.
These proposals define the set of candidate detections available
to the detector. The second module is a large convolutional
neural network that extracts a fixed-length feature
vector from each region. The third module is a set of classspecific
linear SVMs. In this section, the author present their design
decisions for each module, describe their test-time usage,
detail how their parameters are learned, and show detection
results on PASCAL VOC 2010-12 and on ILSVRC2013.

\section{Conclusion}
The author achieved this performance through two insights. The
first is to apply high-capacity convolutional neural networks
to bottom-up region proposals in order to localize
and segment objects. The second is a paradigm for training
large CNNs when labeled training data is scarce. They
show that it is highly effective to pre-train the network��
with supervision��for a auxiliary task with abundant data
(image classification) and then to fine-tune the network for
the target task where data is scarce (detection). They conjecture
that the ��supervised pre-training/domain-specific finetuning��
paradigm will be highly effective for a variety of
data-scarce vision problems.



{\small
\bibliographystyle{ieee}
\bibliography{1}
}

\end{document}
