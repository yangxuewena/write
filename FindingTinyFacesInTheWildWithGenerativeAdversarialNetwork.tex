\documentclass[10pt,twocolumn,letterpaper]{article}

\usepackage{cvpr}
\usepackage{times}
\usepackage{epsfig}
\usepackage{graphicx}
\usepackage{amsmath}
\usepackage{amssymb}

\usepackage{times}


\cvprfinalcopy % *** Uncomment this line for the final submission
\def\cvprPaperID{****} % *** Enter the CVPR Paper ID here
\def\httilde{\mbox{\tt\raisebox{-.5ex}{\symbol{126}}}}

\usepackage{indentfirst}
\setlength{\parindent}{2em}
\usepackage{cite}
\usepackage[colorlinks,linkcolor=red,anchorcolor=blue,citecolor=green,backref=page]{hyperref}

\author{Xuewen Yang\\\\
July 24 2018}

\title{Finding Tiny Faces in the Wild with Generative Adversarial Network}
\begin{document}
\maketitle
\begin{abstract}
Face detection techniques have been developed for
decades, and one of remaining open challenges is detect-
ing small faces in unconstrained conditions. The reason is
that tiny faces are often lacking detailed information and
blurring. In this paper,the author proposed an algorithm to direct-
ly generate a clear high-resolution face from a blurry small
one by adopting a generative adversarial network (GAN).
Toward this end, the basic GAN formulation achieves it by
super-resolving and refining sequentially (e.g. SR-GAN and
cycle-GAN). However,they design a novel network to address
the problem of super-resolving and refining jointly.They also
introduce new training losses to guide the generator net-
work to recover fine details and to promote the discrimina-
tor network to distinguish real vs. fake and face vs. non-face
simultaneously. Extensive experiments on the challenging
dataset WIDER FACE demonstrate the effectiveness of this
proposed method in restoring a clear high-resolution face
from a blurry small one, and show that the detection perfor-
mance outperforms other state-of-the-art methods.
\end{abstract}
\section{Introduction}
Face detection is a fundamental and important problem
in computer vision, since it is usually a key step towards
many subsequent face-related applications, including
face parsing, face verification, face tagging and retrieval,
etc.Modern face detectors have achieved impressive results
on the large and medium faces, however, the performance
on small faces is far from satisfactory. The main difficulty
for small face detection is that small
faces lack sufficient detailed information to distinguish
them from the similar background,To detect small faces, \cite{Xu2017Learning} directly
up-samples images using bi-linear operation and exhaustively
searches faces on the up-sampled images. However,
this method will increase the computation cost and the
inference time will increase significantly too. Moreover,
images are often zoomed in with a small upscaling factors in~\cite{Xu2017Learning}, otherwise, artifacts will be generated.
Besides,it use the intermediate conv feature
maps to represent faces at specific scales, which keeps
the balance between the computation burden and the performance.
However, the shallow but fine-grained intermediate
conv feature maps lack discrimination, which causes many
false positive results. More importantly, these methods take
no care of other challenges, like blur and illumination.
\section{Proposed Method}
In this section,the author introduce our proposed method in details.
First,they give a brief description on the classical GAN
network. Then, the whole architecture of our method is presented. Finally,they introduce each
part of the network in details and define the loss functions
for training the generator network and discriminator network
respectively.
\subsection{GAN}
GAN~\cite{Goodfellow2014Generative} learns a generative model via an adversarial
process. It trains a generator network and a discriminator
network simultaneously. The training process alternately
optimizes the generator and discriminator, which compete
with each other. The generator is trained for generating
the samples to fool the discriminator, and the discriminator is trained to distinguish the real image and the fake
image from the generator.
\subsection{Network Architecture}
Their generator network includes two components and the
first sub-network takes the low-resolution images as the inputs
and the outputs are the super-resolution images. Since
the blurry small faces lack fine details and due to the influence
of MSE loss, the generated super-resolution
faces are usually blurring. So they design the second subnetwork
to refine the super-resolution images from the first
sub-network. Furthermore,they add the classification branch
to the discriminator network for the purpose of detection,
which means their discriminator can classify faces and nonfaces
as well as distinguish the fake and real images.
\subsection{Loss Function}
The author adopt the pixel-wise loss and adversarial loss from
some state-of-the-art approaches to optimize their
generator network. In contrast,the author remove the VGG
feature matching loss due to the calculation cost and we introduce
the classification loss to drive the generator network
to recover fine details from the blurry small faces.
{\small
\bibliographystyle{ieee}
\bibliography{1}
}

\end{document}
