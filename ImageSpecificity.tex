\documentclass[10pt,twocolumn,letterpaper]{article}

\usepackage{cvpr}
\usepackage{times}
\usepackage{epsfig}
\usepackage{graphicx}
\usepackage{amsmath}
\usepackage{amssymb}

\usepackage{times}


\cvprfinalcopy % *** Uncomment this line for the final submission
\def\cvprPaperID{****} % *** Enter the CVPR Paper ID here
\def\httilde{\mbox{\tt\raisebox{-.5ex}{\symbol{126}}}}

\usepackage{indentfirst}
\setlength{\parindent}{2em}
\usepackage{cite}
\usepackage[colorlinks,linkcolor=red,anchorcolor=blue,citecolor=green,backref=page]{hyperref}

\author{Xuewen Yang\\\\
August 23 2018}

\title{Deformable Part Models are Convolutional Neural Networks}
\begin{document}
\maketitle
\begin{abstract}
For some images, descriptions written by multiple people
are consistent with each other. But for other images, descriptions
across people vary considerably. In other words,
some images are specific �C they elicit consistent descriptions
from different people �C while other images are ambiguous.
Applications involving images and text can benefit from an
understanding of which images are specific and which ones
are ambiguous. For instance, consider text-based image retrieval.
If a query description is moderately similar to the
caption (or reference description) of an ambiguous image,
that query may be considered a decent match to the image.
But if the image is very specific, a moderate similarity between
the query and the reference description may not be
sufficient to retrieve the image.
In this paper, The author introduce the notion of image specificity.
We present two mechanisms to measure specificity
given multiple descriptions of an image: an automated measure
and a measure that relies on human judgement. They
analyze image specificity with respect to image content and
properties to better understand what makes an image specific.
They then train models to automatically predict the
specificity of an image from image features alone without
requiring textual descriptions of the image. Finally, they
show that modeling image specificity leads to improvements
in a text-based image retrieval application.
\end{abstract}
\section{Introduction}
Consider the two photographs in Figure~\ref{fig:1}. How would
you describe them? For the first, phrases like ��people lined
up in terminal��, ��people lined up at train station��, ��people
waiting for train outside a station��, etc. come to mind. It is
clear what to focus on and describe. In fact, different people
talk about similar aspects of the image �C the train, people,
station or terminal, lining or queuing up. But for the photograph
on the right, it is less clear how it should be described.
Some people talk about the the sunbeam shining through the
skylight, while others talk about the alleyway, or the people
selling products and walking. In other words, the photograph
on the left is specific whereas the photograph on the
right is ambiguous.
\begin{figure}
\centering
\includegraphics[width=8cm,height=6cm]{1}
\caption{Some images are specific �C they elicit consistent descriptions
from different people (left). Other images (right) are
ambiguous.}
\label{fig:1}
\end{figure}
The computer vision community has made tremendous
progress on recognition problems such as object detection, image classification\cite{Krizhevsky2012ImageNet}, attribute classification and scene recognition\cite{Xiao2010SUN}. Various approaches
are moving to higher-level semantic image understanding
tasks. One such task that is receiving increased
attention in recent years is that of automatically generating
textual descriptions of images and evaluating these descriptions. However, these works have largely
ignored the variance in descriptions produced by different
people describing each image. In fact, early works
that tackled the image description problem\cite{Sadeghi2010Every} or reasoned
about what image content is important and frequently described claimed that human descriptions are consistent.
They show that there is in fact variance in how consistent multiple
human-provided descriptions of the same image are.
Instead of treating this variance as noise, we think of it as
a useful signal that if modeled, can benefit applications involving
images and text.

They introduce the notion of image specificity which measures
the amount of variance in multiple viable descriptions
of the same image. Modeling image specificity can
benefit a variety of applications. For example, computergenerated
image description and evaluation approaches can
benefit from specificity. If an image is known to be ambiguous, several different descriptions can be generated and
be considered to be plausible. But if an image is specific, a
narrower range of descriptions may be appropriate. Photographers,
editors, graphics designers, etc. may want to pick
specific images �C images that are likely to have a single (intended)
interpretation across viewers.


{\small
\bibliographystyle{ieee}
\bibliography{1}
}

\end{document}
