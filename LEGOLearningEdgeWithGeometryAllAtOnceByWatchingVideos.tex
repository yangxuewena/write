\documentclass[10pt,twocolumn,letterpaper]{article}

\usepackage{cvpr}
\usepackage{times}
\usepackage{epsfig}
\usepackage{graphicx}
\usepackage{amsmath}
\usepackage{amssymb}

\usepackage{times}


\cvprfinalcopy % *** Uncomment this line for the final submission
\def\cvprPaperID{****} % *** Enter the CVPR Paper ID here
\def\httilde{\mbox{\tt\raisebox{-.5ex}{\symbol{126}}}}

\usepackage{indentfirst}
\setlength{\parindent}{2em}
\usepackage{cite}
\usepackage[colorlinks,linkcolor=red,anchorcolor=blue,citecolor=green,backref=page]{hyperref}

\author{Xuewen Yang\\\\
July 26 2018}

\title{Recurrent Convolutional Neural Network for Object Recognition}
\begin{document}
\maketitle
\begin{abstract}
Learning to estimate 3D geometry in a single image by
watching unlabeled videos via deep convolutional network
is attracting significant attention. In this paper,the author introduce
a ��3D as-smooth-as-possible (3D-ASAP)�� prior inside
the pipeline, which enables joint estimation of edges
and 3D scene, yielding results with significant improvement
in accuracy for fine detailed structures. Specifically,
they define the 3D-ASAP prior by requiring that any two
points recovered in 3D from an image should lie on an existing
planar surface if no other cues provided.They design
an unsupervised framework that Learns Edges and Geometry
(depth, normal) all at Once (LEGO). The predicted
edges are embedded into depth and surface normal smoothness
terms, where pixels without edges in-between are constrained
to satisfy the prior. In their framework, the predicted
depths, normals and edges are forced to be consistent all the
time.
\end{abstract}
\section{Introduction}
Recently, impressive progress\cite{Godard2016Unsupervised}\cite{Zhou2017Unsupervised} has been made
to mimic detailed 3D reconstruction by training a deep network
taking only unlabeled videos or stereo images as input
and testing on monocular image, yielding even better depth
estimation results than those of supervised methods\cite{Eigen2014Depth}
in outdoor scenarios. The core underlying idea is the supervision
by view synthesis, where the frame of one view
(source) is warped to another (target) based on the predicted

\section{Related Work}
Geometric based methods estimate 3D from a given video with
feature matching, such as SFM, SLAM and
DTAM, which could be effective and efficient in many
cases.Deep neural networks (DCN) developed in recent years,
e.g.VGG and ResNet, provide strong feature representation.
Dense geometry, i.e., pixel-wise depth and
normal maps, can be readily estimated from a single image\cite{Eigen2015Predicting}. The learned CNN model shows
significant improvement compared to other methods based
on hand-crafted features.Motivated by traditional
methods, videos, which are easier to obtain and
hold richer 3D information. Motivated by traditional methods
like SFM and DTAM, lots of CNN based methods are
proposed to do single view geometry estimation with supervision
from vieos, and yield impressive progress.Long range and non-local spatial
regularization has been vastly explored in classical
graphical models like CRF\cite{Lafferty2001Conditional}, where nodes beyond the
neighboring are connected, and the smoothness in-between
are learned with high-order CRF or densely-connected
CRF.

\section{Conclusion}
In this paper,the author proposed LEGO, an unsupervised
framework for joint depth, normal and edge learning. A
novel 3D-ASAP prior is proposed to better regularize the
learning of scene layout. This regularization jointly considers
the three important descriptors of 3D scene and improves
the results on all tasks: depth, normal and edge
estimation.They conducted comprehensive experiments to
present the performance of LEGO. On KITTI dataset,
LEGO achieves SOTA performance on both depth and normal
evaluation. For edge evaluation, LEGO outperformes
the other methods by a large margin on Cityscapes dataset.
{\small
\bibliographystyle{ieee}
\bibliography{1}
}

\end{document}
