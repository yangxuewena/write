\documentclass[10pt,twocolumn,letterpaper]{article}

\usepackage{cvpr}
\usepackage{times}
\usepackage{epsfig}
\usepackage{graphicx}
\usepackage{amsmath}
\usepackage{amssymb}

\usepackage{times}


\cvprfinalcopy % *** Uncomment this line for the final submission
\def\cvprPaperID{****} % *** Enter the CVPR Paper ID here
\def\httilde{\mbox{\tt\raisebox{-.5ex}{\symbol{126}}}}

\usepackage{indentfirst}
\setlength{\parindent}{2em}
\usepackage{cite}
\usepackage[colorlinks,linkcolor=red,anchorcolor=blue,citecolor=green,backref=page]{hyperref}

\author{Xuewen Yang\\\\
August 11 2018}

\title{Embodied Question Answering}
\begin{document}
\maketitle
\begin{abstract}
The author present a new AI task �C Embodied Question Answering
(EmbodiedQA) �C where an agent is spawned at a random
location in a 3D environment and asked a question (��What
color is the car?��). In order to answer, the agent must first intelligently
navigate to explore the environment, gather necessary
visual information through first-person (egocentric)
vision, and then answer the question (��orange��).
EmbodiedQA requires a range of AI skills �C language understanding,
visual recognition, active perception, goaldriven
navigation, commonsense reasoning, long-term
memory, and grounding language into actions. In this work,
the author develop a dataset of questions and answers in House3D
environments, evaluation metrics, and a hierarchical
model trained with imitation and reinforcement learning.
\end{abstract}
\section{EQA Dataset: Questions In Environments}
\subsection{House3D: Simulated 3D Environments}
The author instantiate EmbodiedQA in House3D, a recently introduced
rich, simulated environment based on 3D indoor
scenes from the SUNCG dataset\cite{Song2016Semantic}. Concretely, SUNCG
consists of synthetic 3D scenes with realistic room and furniture
layouts, manually designed and crowdsourced using
an online interior design interface. Each
scene was further verified as realistic by majority vote of 3
human annotators. In total, SUNCG contains over 45k environments
with 49k valid floors, 404k rooms containing 5
million object instances of 2644 unique objects from 80 different
categories. House3D converts SUNCG from a static
3D dataset to a set of simulated environments, where an
agent (approximated as a 1 meter high cylinder) may navigate
under simple physical constraints (not being able to
pass through walls/objects).
\subsection{QuestionAnswer Generation}
The author draw inspiration from the CLEVR\cite{Johnson2016CLEVR} dataset, and programmatically
generate a dataset (EQA) of grounded questions
and answers. This gives the ability to carefully control
the distribution of question-types and answers in the
dataset, and deter algorithms from exploiting dataset bias.

\section{Imitation Learning and Reward Shaping}
The navigation module is trained to mimic the shortest path
in a teacher-forcing setting - i.e., given the navigation history
encoded in the hidden state of the planner, question encoding,
and the current frame, the model is trained to predict
the action that would keep it on the shortest path. The author use
a cross-entropy loss and train the model for 15 epochs. They
find that even in this imitation learning case, it is essential
to train the model under a distance-based curriculum. In the
first epoch, the author backtrack 10 steps from the target along the
shortest path and initialize the agent at this point with full
history of the trajectory from spawned location. They step
back an additional 10 steps at each successive epoch. They
train for 15 epochs total with batch size ranging from 5 to
20 (depending on path length due to memory limitations).
The answering module is trained to predict the correct answer
from the question and shortest path navigation frames,
with cross-entropy loss for 50 epochs and batch size of 20.
{\small
\bibliographystyle{ieee}
\bibliography{1}
}

\end{document}
