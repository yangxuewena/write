\documentclass[10pt,twocolumn,letterpaper]{article}

\usepackage{cvpr}
\usepackage{times}
\usepackage{epsfig}
\usepackage{graphicx}
\usepackage{amsmath}
\usepackage{amssymb}

\usepackage{times}


\cvprfinalcopy % *** Uncomment this line for the final submission
\def\cvprPaperID{****} % *** Enter the CVPR Paper ID here
\def\httilde{\mbox{\tt\raisebox{-.5ex}{\symbol{126}}}}

\usepackage{indentfirst}
\setlength{\parindent}{2em}
\usepackage{cite}
\usepackage[colorlinks,linkcolor=red,anchorcolor=blue,citecolor=green,backref=page]{hyperref}

\author{Xuewen Yang\\\\
July 20 2018}

\title{Deep Depth Completion of a Single RGB-D Image}
\begin{document}
\maketitle
\begin{abstract}
The goal of this work is to complete the depth channel of
an RGB-D image. Commodity-grade depth cameras often
fail to sense depth for shiny, bright, transparent, and distant
surfaces. To address this problem, the author train a deep network
that takes an RGB image as input and predicts dense surface
normals and occlusion boundaries. Those predictions
are then combined with raw depth observations provided by
the RGB-D camera to solve for depths for all pixels, including
those missing in the original observation. This method
was chosen over others (e.g., inpainting depths directly) as
the result of extensive experiments with a new depth completion
benchmark dataset, where holes are filled in training
data through the rendering of surface reconstructions created
from multiview RGB-D scans. Experiments with different
network inputs, depth representations, loss functions,
optimization methods, inpainting methods, and deep depth
estimation networks show that our proposed approach provides
better depth completions than these alternatives.
\end{abstract}
\section{Introduction}
Depth sensing has become pervasive in applications
as diverse as autonomous driving, augmented reality, and
scene reconstruction. Despite recent advances in depth
sensing technology, commodity-level RGB-D cameras like
Microsoft Kinect, Intel RealSense, and Google Tango still
produce depth images with missing data when surfaces are
too glossy, bright, thin, close, or far from the camera. These
problems appear when rooms are large, surfaces are shiny,
and strong lighting is abundant �C e.g., in museums, hospitals,
classrooms, stores, etc. Even in homes, depth images
often are missing more than 50\% of the pixels (Figure~\ref{fig:1}).
\begin{figure}
\centering
\includegraphics[width=8cm,height=6cm]{1}
\caption{Using surface normals to solve for depth completion.
(a) An example of where depth cannot be solved from surface normal.
(b) The area missing depth is marked in red. The red arrow
shows paths on which depth cannot be integrated from surface normals.
However in real-world images, there are usually many paths
through connected neighboring pixels (along floors, ceilings, etc.)
over which depths can be integrated (green arrows).}
\label{fig:1}
\end{figure}
\section{Related Work}
There has been a large amount of prior work on depth estimation, inpainting, and processing.

Depth estimation from a monocular
color image is a long-standing problem in computer vision.
Classic methods include shape-from-shading and
shape-from-defocus\cite{Suwajanakorn2015Depth}. Other early methods were based
on hand-tuned models and/or assumptions about surface
orientations\cite{Saxena2009Make3D}. Newer methods treat depth estimation
as a machine learning problem, most recently using
deep networks. For example, Eigen et al. first used a multiscale convolutional network to regress from color
images to depths\cite{Eigen2015Predicting}\cite{Eigen2014Depth}.

Many methods have been proposed for
filling holes in depth channels of RGB-D images, including
ones that employ smoothness priors, fast marching
methods, Navier-Stokes, anisotropic diffusion, background surface extrapolation,
color-depth edge alignment, low-rank matrix
completion, tensor voting, Mumford-Shah functional
optimization, joint optimization with other properties
of intrinsic images, and patch-based image synthesis.
\section{Method}
In this paper,the author investigate how to use a deep network to
complete the depth channel of a single RGB-D image.The
investigation focuses on the following questions: ��how can
they get training data for depth completion?,�� ��what depth
representation should they use?,�� and ��how should cues from
color and depth be combined?.��
\subsection{Dateset}
To create the dataset,the author utilize existing surface
meshes reconstructed from multi-view RGB-D scans
of large environments. There are several datasets of this
type, including Matterport3D, ScanNet, SceneNN, and SUN3D, to name a few.They use Matterport3D.
For each scene,they extract a triangle mesh M with
x1-6 million triangles per room from a global surface reconstruction
using screened Poisson surface reconstruction. Then, for a sampling of RGB-D images in the scene,
the author render the reconstructed mesh M from the camera pose
of the image viewpoint to acquire a completed depth image.
\subsection{Depth Representation}
The author focus on predicting surface normals and
occlusion boundaries. Since normals are differential surface
properties, they depend only on local neighborhoods
of pixels. Moreover, they relate strongly to local lighting
variations directly observable in a color image. For
these reasons, previous works on dense prediction of surface
normals from color images produce excellent results\cite{Eigen2015Predicting}. Similarly, occlusion boundaries produce
local patterns in pixels (e.g., edges), and so they usually can
be robustly detected with a deep network.


{\small
\bibliographystyle{ieee}
\bibliography{1}
}

\end{document}
