\documentclass[10pt,twocolumn,letterpaper]{article}

\usepackage{cvpr}
\usepackage{times}
\usepackage{epsfig}
\usepackage{graphicx}
\usepackage{amsmath}
\usepackage{amssymb}

\usepackage{times}


\cvprfinalcopy % *** Uncomment this line for the final submission
\def\cvprPaperID{****} % *** Enter the CVPR Paper ID here
\def\httilde{\mbox{\tt\raisebox{-.5ex}{\symbol{126}}}}

\usepackage{indentfirst}
\setlength{\parindent}{2em}
\usepackage{cite}
\usepackage[colorlinks,linkcolor=red,anchorcolor=blue,citecolor=green,backref=page]{hyperref}

\author{Xuewen Yang\\\\
August 8 2018}

\title{Deep Neural Networks are Easily Fooled:High Confidence Predictions for Unrecognizable Images}
\begin{document}
\maketitle
\begin{abstract}
Deep neural networks (DNNs) have recently been
achieving state-of-the-art performance on a variety of
pattern-recognition tasks, most notably visual classification
problems. Given that DNNs are now able to classify objects
in images with near-human-level performance, questions
naturally arise as to what differences remain between computer
and human vision. A recent study revealed that
changing an image (e.g. of a lion) in a way imperceptible to
humans can cause a DNN to label the image as something
else entirely (e.g. mislabeling a lion a library). Here the author
show a related result: it is easy to produce images that are
completely unrecognizable to humans, but that state-of-theart
DNNs believe to be recognizable objects with 99.99\%
confidence (e.g. labeling with certainty that white noise
static is a lion). Specifically, they take convolutional neural
networks trained to perform well on either the ImageNet
or MNIST datasets and then find images with evolutionary
algorithms or gradient ascent that DNNs label with high
confidence as belonging to each dataset class. It is possible
to produce images totally unrecognizable to human eyes
that DNNs believe with near certainty are familiar objects,
which they call ��fooling images�� (more generally, fooling examples).
Their results shed light on interesting differences
between human vision and current DNNs, and raise questions
about the generality of DNN computer vision.
\end{abstract}
\section{Introduction}
Deep neural networks (DNNs) learn hierarchical layers
of representation from sensory input in order to perform
pattern recognition\cite{Hinton2007Learning}. Recently, these deep architectures
have demonstrated impressive, state-of-the-art,
and sometimes human-competitive results on many pattern
recognition tasks, especially vision classification problems\cite{Le2011Learning}. Given the near-human ability of DNNs to
classify visual objects, questions arise as to what differences
remain between computer and human vision.

A recent study revealed a major difference between DNN
and human vision. Changing an image, originally correctly
classified (e.g. as a lion), in a way imperceptible to
human eyes, can cause a DNN to label the image as something
else entirely (e.g. mislabeling a lion a library).
\section{Methods}
\subsection{Deep neural network models}
To test whether DNNs might give false positives for
unrecognizable images, they need a DNN trained to near
state-of-the-art performance. They choose the well-known
��AlexNet�� architecture from\cite{Krizhevsky2012ImageNet}, which is a convnet
trained on the 1.3-million-image ILSVRC 2012 ImageNet
dataset. Specifically, they use the already-trained
AlexNet DNN provided by the Caffe software package.
It obtains 42.6\% top-1 error rate, similar to the 40.7\% reported
by Krizhevsky 2012\cite{Krizhevsky2012ImageNet}. While the Caffe-provided
DNN has some small differences from Krizhevsky 2012, they do not believe our results would be qualitatively
changed by small architectural and optimization differences
or their resulting small performance improvements. Similarly,
while recent papers have improved upon Krizhevsky
2012, those differences are unlikely to change our results.
We chose AlexNet because it is widely known and a trained DNN similar to it is publicly available. In this paper, they
refer to this model as ��ImageNet DNN��.
\subsection{Generating images with evolution}
They test EAs with two different encodings, meaning
how an image is represented as a genome. The first
has a direct encoding, which has one grayscale integer for
each of 28$\times$28 pixels for MNIST, and three integers (H, S,
V) for each of 256$\times$256 pixels for ImageNet. Each pixel
value is initialized with uniform random noise within the
[0; 255] range. Those numbers are independently mutated;
first by determining which numbers are mutated, via a rate
that starts at 0.1 (each number has a 10\% chance of being
chosen to be mutated) and drops by half every 1000 generations.
The numbers chosen to be mutated are then altered
via the polynomial mutation operator with a fixed mutation
strength of 15. The second EA has an indirect encoding,
which is more likely to produce regular images, meaning
images that contain compressible patterns (e.g. symmetry
and repetition). Indirectly encoded images tend to
be regular because elements in the genome can affect multiple
parts of the image. Specifically, the indirect encoding
here is a compositional pattern-producing network
(CPPN), which can evolve complex, regular images that resemble
natural and man-made objects.
{\small
\bibliographystyle{ieee}
\bibliography{1}
}

\end{document}
