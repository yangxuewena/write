\documentclass[10pt,twocolumn,letterpaper]{article}

\usepackage{cvpr}
\usepackage{times}
\usepackage{epsfig}
\usepackage{graphicx}
\usepackage{amsmath}
\usepackage{amssymb}

\usepackage{times}


\cvprfinalcopy % *** Uncomment this line for the final submission
\def\cvprPaperID{****} % *** Enter the CVPR Paper ID here
\def\httilde{\mbox{\tt\raisebox{-.5ex}{\symbol{126}}}}

\usepackage{indentfirst}
\setlength{\parindent}{2em}
\usepackage{cite}
\usepackage[colorlinks,linkcolor=red,anchorcolor=blue,citecolor=green,backref=page]{hyperref}

\author{Xuewen Yang\\\\
August 21 2018}

\title{Deformable Part Models are Convolutional Neural Networks}
\begin{document}
\maketitle
\section{Object Localization Module}
The Scale-Transferrable Detection Network (STDN)
consists of a base network and two task specific prediction
subnetworks. The role of the base network is to do feature extraction. The first subnet is used for object classification,
and the second subnet is used for bounding box position
regression.
\subsection{Anchor Boxes.}
The author associate a set of default anchor boxes with each feature
map which is got by scale-transfer module. The scale
of anchor boxes is the same as that of SSD\cite{Liu2015SSD}. Following
DSSD, they use [1:6; 2:0; 3:0] aspect ratios at every prediction
layer. Anchors are matched to any ground truth with
intersection-over-union (IoU) higher than a threshold (0:5).
The remaining anchors are treated as background. After the
matching step, most of the default anchor boxes are negatives
(no matched). They use hard negative mining so that the
ratio between the negatives and positives is at most 3 : 1.
\subsection{Classification Subnet.}
The role of the classification subnet is to predict the probability
of each anchor belonging to a category. It contains
a 1$\times$1 convolution layer and two 3$\times$3 convolution layers.
Each convolution layer has a batchnorm layer\cite{Ioffe2015Batch} and
a relu layer in front of it. The last convolution layer has K A
filters, where K is the number of object classes and A is the
number of anchors per spatial location. The classification
loss is the softmax loss over multiple classes confidences.
\subsection{Box Regression Subnet.}
The purpose of this subnet is to regress the offset from each
anchor box to the matched ground-truth object. The box
regression subnet has the same structure as the classification
subnet except that its last convolution layer has 4A filters.
Smooth L1 loss is used for the localization loss and the
bounding box loss is only used for positive samples
\subsection{Training Settings.}
Our detector is based on the MXNet\cite{Gidaris2015Object} framework. All our
models are trained with SGD solver on NVIDIA TITAN Xp
GPU. We follow almost the same training strategy as SSD\cite{Liu2015SSD}, including a random expansion data augmentation trick
which is helpful for detecting small objects.

{\small
\bibliographystyle{ieee}
\bibliography{1}
}

\end{document}
