\documentclass[10pt,twocolumn,letterpaper]{article}

\usepackage{cvpr}
\usepackage{times}
\usepackage{epsfig}
\usepackage{graphicx}
\usepackage{amsmath}
\usepackage{amssymb}

\usepackage{times}


\cvprfinalcopy % *** Uncomment this line for the final submission
\def\cvprPaperID{****} % *** Enter the CVPR Paper ID here
\def\httilde{\mbox{\tt\raisebox{-.5ex}{\symbol{126}}}}

\usepackage{indentfirst}
\setlength{\parindent}{2em}
\usepackage{cite}
\usepackage[colorlinks,linkcolor=red,anchorcolor=blue,citecolor=green,backref=page]{hyperref}

\author{Xuewen Yang\\\\
August 15 2018}

\title{Deformable Part Models are Convolutional Neural Networks}
\begin{document}
\maketitle
\section{Implementation details}
The author implemented their experiments by modifying the
DPM voc-release code and using Caffe for CNN
computation.They describe the most important implementation
details. Source code for the complete
system is available, thus providing documentation of the remaining
implementation details.
\subsection{Parameter learning}
There are at least two natural ways to train a DeepPyramid
DPM. The first treats the model as a single CNN and
trains it end-to-end with SGD and backpropagation. The
second trains the model in two stages: (1) fit the front-end
CNN; (2) train a DPM on top of stage 1 using latent SVM\cite{Everingham2010The} while keeping the front-end CNN fixed. The first procedure
is more in the spirit of deep learning, while the second
is an important baseline for comparison to traditional
HOG-based DPM and for showing if end-to-end training is
useful.They chose to focus on latent SVM training since it is
a necessary baseline for end-to-end training. Although we
don��t explore end-to-end training due to space constraints,
they point interested readers to contemporaneous work by
Wan~\emph{et al.} that shows results of a similar model trained
end-to-end. They report modest improvements from endto-
end optimization.
\subsection{Feature pyramid construction}
Any fully-convolutional network can be used to generate
the feature pyramid. In this DeepPyramid DPM implementation,they chose to use a truncated variant of the SuperVision
CNN\cite{Krizhevsky2012ImageNet}. In order to directly compare our results
with R-CNN, they use the publicly available network
weights that are distributed with R-CNN. These weights
were trained on the ILSVRC 2012 classification training
dataset using Caffe (they do not use the detection fine-tuned
weights since they were trained on warped image windows).
\subsection{DPM training and testing details}
Compared to training a DPM with HOG features, they
found it necessary to make some changes to the standard
DPM training procedure. First, they don��t use left/right mirrored
pairs of mixture components. These components are
easy to implement with HOG because model parameters
can be explicitly ��flipped�� allowing them to be tied between
mirrored components. Second, R-CNN and DPM use different
non-maximum suppression functions and they found
that the one used in R-CNN, which is based on intersectionover-
union (IoU) overlap, performs slightly better with
conv5 features (but is worse for the baseline HOG-DPM).
Lastly, they found that it��s very important to use poorly localized
detections of ground-truth objects as negative examples.
As in R-CNN, they define negative examples as all
detections that have a max IoU overlap with a ground-truth
object of less than 0.3. Using poorly localized positives as
negative examples leads to substantially better results than just using negatives from negative images,
which is the standard practice when training HOG-DPM.
Using these difficult negative examples in HOG-DPM did
not improve the baseline results.


{\small
\bibliographystyle{ieee}
\bibliography{1}
}

\end{document}
