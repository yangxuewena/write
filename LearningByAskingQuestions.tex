\documentclass[10pt,twocolumn,letterpaper]{article}

\usepackage{cvpr}
\usepackage{times}
\usepackage{epsfig}
\usepackage{graphicx}
\usepackage{amsmath}
\usepackage{amssymb}

\usepackage{times}


\cvprfinalcopy % *** Uncomment this line for the final submission
\def\cvprPaperID{****} % *** Enter the CVPR Paper ID here
\def\httilde{\mbox{\tt\raisebox{-.5ex}{\symbol{126}}}}

\usepackage{indentfirst}
\setlength{\parindent}{2em}
\usepackage{cite}
\usepackage[colorlinks,linkcolor=red,anchorcolor=blue,citecolor=green,backref=page]{hyperref}

\author{Xuewen Yang\\\\
July 18 2018}

\title{Learning by Asking Questions}
\begin{document}
\maketitle
\begin{abstract}
In this paper,the author introduce an interactive learning framework for the
development and testing of intelligent visual systems, called
learning-by-asking (LBA).They explore LBA in context of the
Visual Question Answering (VQA) task. LBA differs from
standard VQA training in that most questions are not ob-
served during training time, and the learner must ask ques-
tions it wants answers to. Thus, LBA more closely mim-
ics natural learning and has the potential to be more data-
efficient than the traditional VQA setting.They present a
model that performs LBA on the CLEVR dataset, and show
that it automatically discovers an easy-to-hard curriculum
when learning interactively from an oracle. Our LBA gener-
ated data consistently matches or outperforms the CLEVR
train data and is more sample efficient.The author also show that
the model asks questions that generalize to state-of-the-art
VQA models and to novel test time distributions.
\end{abstract}
\section{Introduction}
Machine learning models have led to remarkable
progress in visual recognition. However, while the training
data that is fed into these models is crucially important,
it is typically treated as predetermined, static information.
The current models are passive in nature: they rely on training
data curated by humans and have no control over this
supervision. This is in stark contrast to the way we humans
learn �� by interacting with our environment to gain information.
The interactive nature of human learning makes
it sample efficient (there is less redundancy during training)
and also yields a learning curriculum (we ask for more complex
knowledge as we learn).

In this paper,the author argue that next-generation recognition
systems need to have agency �� the ability to decide what
information they need and how to get it.They explore this in
the context of visual question answering (VQA).
Instead of training on a fixed, large-scale dataset, we propose
an alternative interactive VQA setup called learning-
by-asking (LBA): at training time, the learner receives only images and decides what questions to ask. Questions asked
by the learner are answered by an oracle (human supervision).
At test-time, LBA is evaluated exactly like VQA using
well understood metrics.
\section{Related Work}
Visual question answering (VQA) is a surrogate task
designed to assess a system��s ability to thoroughly understand
images. It has gained popularity in recent years due
to the release of several benchmark datasets\cite{Malinowski2014A}. Motivated
by the well-studied difficulty of analyzing results on
real-world VQA datasets\cite{Jabri2016Revisiting}\cite{Zhang2016Yin}, Johnson et al. [23] recently
proposed a more controlled, synthetic VQA dataset
that we adopt in this work.

Visual question generation (VQG) was recently proposed
as an alternative to image captioning.
Our work is related to VQG in the sense that we require the
learner to generate questions about images, however, our
objective in doing so is different. Whereas VQG focuses on
asking questions that are relevant to the image content, LBA
requires the learner to ask questions that are both relevant
and informative to the learner when answered.

Active learning (AL) involves a collection of unlabeled
examples and a learner that selects which samples will be
labeled by an oracle. Common selection
criteria include entropy\cite{Joshi2009Multi}, boosting the margin for classifiers and expected informativeness.

\section{Approach}
They propose an LBA agent built from three modules: (1) a question proposal module that generates a set of question
proposals for an input image; (2) a question answering
module (or VQA model) that predicts answers from (I, q)
pairs; and (3) a question selection module that looks at
both the answering module��s state and the proposal module��s
questions to pick a single question to ask the oracle.
After receiving the oracle��s answer, the agent creates a tuple
(I, q, a) that is used as the online learning signal for all
three modules. Each of the modules is described in a separate
subsection below; the interactions between them are
illustrated in Figure~\ref{fig:1}.
\begin{figure*}
\centering
\includegraphics[width=18cm,height=6cm]{1}
\caption{The approach to the learning-by-asking setting for VQA. Given an image I, the agent generates a diverse set of questions using a question generator g. It then filters out ��irrelevant�� questions using a relevance model r to produce a list of question proposals. The agent then answers its own questions using the VQA model v. With these predicted answers and its self-knowledge of past performance, it selects one question from the proposals to be answered by the oracle. The oracle provides answer-level supervision from which the agent learns to ask informative questions in subsequent iterations.}
\label{fig:1}
\end{figure*}



{\small
\bibliographystyle{ieee}
\bibliography{1}
}

\end{document}
