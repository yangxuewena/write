\documentclass[10pt,twocolumn,letterpaper]{article}

\usepackage{cvpr}
\usepackage{times}
\usepackage{epsfig}
\usepackage{graphicx}
\usepackage{amsmath}
\usepackage{amssymb}

\usepackage{times}


\cvprfinalcopy % *** Uncomment this line for the final submission
\def\cvprPaperID{****} % *** Enter the CVPR Paper ID here
\def\httilde{\mbox{\tt\raisebox{-.5ex}{\symbol{126}}}}

\usepackage{indentfirst}
\setlength{\parindent}{2em}
\usepackage{cite}
\usepackage[colorlinks,linkcolor=red,anchorcolor=blue,citecolor=green,backref=page]{hyperref}

\author{Xuewen Yang\\\\
July 10 2018}

\title{Learning a Convolutional Neural Network for Non-uniform
Motion Blur Removal}
\begin{document}
\maketitle
\begin{abstract}
In this paper,the author address the problem of estimating and
removing non-uniform motion blur from a single blurry image.
They propose a deep learning approach to predicting
the probabilistic distribution of motion blur at the patch
level using a convolutional neural network (CNN).They further
extend the candidate set of motion kernels predicted
by the CNN using carefully designed image rotations. A
Markov random field model is then used to infer a dense
non-uniform motion blur field enforcing motion smoothness.
Finally, motion blur is removed by a non-uniform deblurring
model using patch-level image prior. Experimental
evaluations show that this approach can effectively estimate
and remove complex non-uniform motion blur that is not
handled well by previous approaches.
\end{abstract}
\section{Introduction}
Image deblurring\cite{Cho2007Removing} aims at
recovering sharp image from a blurry image due to camera
shake, object motion or out-of-focus. In this paper,they focus
on estimating and removing spatially varying motion blur.

Non-uniform deblurring has attracted much
attention in recent years. Methods in\cite{Gupta2010Single}\cite{Hirsch2011Fast} work
on non-uniform blur caused by camera rotations, in-plane
translations or forward out-of-plane translations. They are
effective for removing non-uniform blur consistent with
these motion assumptions. Another category of approaches
works on non-uniform motion blur caused by object motion.
They estimate blur kernels by analyzing image statistics, blur spectrum, or with a learning approach using
hand-crafted features\cite{Couzini2013Learning}.Other approaches
jointly estimate the sharp image and blur kernels using
a sparsity prior. It is still challenging today to remove
strongly non-uniform motion blur captured in complex
scenes.

In this work,they propose a novel deep learning-based approach
to estimating non-uniform motion blur, followed by a patch statistics-based deblurring model adapted to nonuniform
motion blur.They estimate the probabilities of motion
kernels at the patch level using a convolutional neural
network (CNN), then fuse the patch-based
estimations into a dense field of motion kernels using a
Markov random field (MRF) model. To fully utilize the
CNN,they propose to extend the candidate motion kernel
set predicted by CNN using an image rotation technique,
which significantly boost its performance for motion kernel
estimation. Taking advantage of the strong feature learning
power of CNNs,they can well predict the challenging nonuniform
motion blur that can hardly be well estimated by
the state-of-the-art approaches.

Figure~\ref{fig:1} illustrates our approach. Given a blurry image,
we first estimate non-uniform motion blur field by a CNN
model, then we deconvolve the blurry image. Our approach
can effectively estimate the spatially varying motion kernels,
which enable us to well remove the motion blur.
\begin{figure*}
\centering
\includegraphics[width=15cm,height=6cm]{1}
\caption{An example illustrating our approach. Given an image with non-uniform motion blur (left). We first estimate the field of
non-uniform motion blur kernels by a convolutional neural network (middle), then deconvolve the blurred image (right).}
\label{fig:1}
\end{figure*}
\section{RelatedWork}
Estimating accurate motion blur kernels is essential to
non-uniform image deblurring. In\cite{Gupta2010Single}\cite{Hirsch2011Fast}, nonuniform
motion blur is modeled as a global camera motion,
which basically estimates an uniform kernel in the camera
motion space. Methods jointly estimate the
motion kernels and sharp image. They rely on a sparsity
prior to infer the latent sharp image for better motion kernel
estimation. Different to them, we estimate motion blur kernels
directly using the local patches, which does not require
the estimation of camera motion or a latent sharp image.

Another category of approaches estimates spatially
varying motion blur based on local image features. The
method estimates motion blur based on blur spectrum
analysis of image patch in Fourier transform space.\cite{B2006Blind}
predicts motion blur kernel using natural image statistics. estimates motion blur by analyzing the alpha maps of
image edges.\cite{Couzini2013Learning} learns a regression function to predict motion
blur kernel based on some hand-crafted features. Different
to them, we estimate motion blur kernels using a convolutional
neural network, followed by a carefully designed
motion kernel extension method and MRF model to predict a dense field of motion kernels.This approach can well estimate
complex and strong motion blur, which can hardly be
well estimated by the previous approaches.
{\small
\bibliographystyle{ieee}
\bibliography{1}
}

\end{document}
