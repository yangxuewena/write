\documentclass[10pt,twocolumn,letterpaper]{article}

\usepackage{cvpr}
\usepackage{times}
\usepackage{epsfig}
\usepackage{graphicx}
\usepackage{amsmath}
\usepackage{amssymb}

\usepackage{times}


\cvprfinalcopy % *** Uncomment this line for the final submission
\def\cvprPaperID{****} % *** Enter the CVPR Paper ID here
\def\httilde{\mbox{\tt\raisebox{-.5ex}{\symbol{126}}}}

\usepackage{indentfirst}
\setlength{\parindent}{2em}
\usepackage{cite}
\usepackage[colorlinks,linkcolor=red,anchorcolor=blue,citecolor=green,backref=page]{hyperref}

\author{Xuewen Yang\\\\
August 13 2018}

\title{Deformable Part Models are Convolutional Neural Networks}
\begin{document}
\maketitle
\begin{abstract}
Deformable part models (DPMs) and convolutional neural
networks (CNNs) are two widely used tools for visual
recognition. They are typically viewed as distinct approaches:
DPMs are graphical models (Markov random
fields), while CNNs are ��black-box�� non-linear classifiers.
In this paper, the authro show that a DPM can be formulated as a
CNN, thus providing a synthesis of the two ideas. Our construction
involves unrolling the DPM inference algorithm
and mapping each step to an equivalent CNN layer. From
this perspective, it is natural to replace the standard image
features used in DPMs with a learned feature extractor.
We call the resulting model a DeepPyramid DPM and experimentally
validate it on PASCAL VOC object detection.
We find that DeepPyramid DPMs significantly outperform
DPMs based on histograms of oriented gradients features
(HOG) and slightly outperforms a comparable version of
the recently introduced R-CNN detection system, while running
significantly faster.
\end{abstract}
\section{Introduction}
Part-based representations are widely used in visual
recognition. In particular, deformable part models (DPMs) have been effective for generic object category detection.
DPMs update pictorial structure models\cite{Felzenszwalb2005Pictorial}, which
date back to the 1970s, with modern image features
and machine learning algorithms.

Convolutional neural networks (CNNs) are another influential
class of models for visual recognition. CNNs also
have a long history, and have resurged over the
last two years due to good performance on image classification, object detection, and more recently a wide
variety of vision tasks (e.g.,\cite{Hariharan2014Simultaneous}\cite{Tompson2014Joint}).

These two models, DPMs and CNNs, are typically
viewed as distinct approaches to visual recognition. DPMs
are graphical models (Markov random fields), while CNNs
are ��black-box�� non-linear classifiers. Many people ask:
Are these models actually distinct? To answer this question
the author show that any DPM can be formulated as an equivalent CNN. In other words, deformable part models are convolutional
neural networks. This construction relies on a new
network layer, distance transform pooling, which generalizes
max pooling.

DPMs typically operate on a scale-space pyramid of gradient
orientation feature maps (HOG\cite{Dalal2005Histograms}). But the author now
know that for object detection this feature representation is
suboptimal compared to features computed by deep convolutional
networks. As a second innovation, they replace
HOG with features learned by a fully-convolutional
network. This ��front-end�� network generates a pyramid of
deep features, analogous to a HOG feature pyramid. They 
call the full model a DeepPyramid DPM.
\section{DeepPyramid DPMs}
A DeepPyramid DPM is a convolutional network that
takes as input an image pyramid and produces as output a
pyramid of object detection scores. Although the model is
a single network, for pedagogical reasons we describe it in
terms of two smaller networks, a feature pyramid ��frontend��
CNN and a DPM-CNN��their function composition
yields the full network. A schematic diagram of the model
is presented in Figure~\ref{fig:1}.
\subsection{Feature pyramid frontend CNN}
Objects appear at all scales in images. A standard technique
for coping with this fact is to run a detector at multiple
scales using an image pyramid. In the context of CNNs, this
method dates back to (at least) early work on face detection, and has been used again in contemporary works, including
OverFeat, DetectorNet, DenseNet, and
SPP-net, a recently proposed method for speeding up
R-CNNs. We follow this approach and use as our front-end
CNN a network that maps an image pyramid to a feature
pyramid. To do this, we use a standard single-scale architecture
(Krizhevsky~\emph{et al.}) and tie the network weights
across all scales.
\subsection{Constructing an equivalent CNN from a DPM}
In the DPM formalism, an object class is modeled as a
mixture of ��components��, each being responsible for modeling
the appearance of an object sub-category (e.g., side
views of cars, people doing handstands, bi-wing propeller
planes). Each component, in turn, uses a low-resolution
global appearance model of the sub-type (called a ��root filter��),
together with a small number of higher resolution
��part filters�� that capture the appearance of local regions
of the sub-type.
\begin{figure*}
\centering
\includegraphics{1}
\caption{Schematic model overview. (1) An image pyramid is built from a color input image. (2) Each pyramid level is forward
propagated through a fully-convolutional CNN (e.g., a truncated SuperVision CNN [27] that ends at convolutional layer 5). (3) The result
is a pyramid of conv5 feature maps, each at 1/16th the spatial resolution of its corresponding image pyramid level. (4) Each conv5 level is
then input into a DPM-CNN, which (5) produces a pyramid of DPM detection scores. Since the whole system is the composition of two
CNNs, it can be viewed as a single, unified CNN that takes a color image pyramid as input and outputs a DPM score pyramid.}
\label{fig:1}
\end{figure*}

{\small
\bibliographystyle{ieee}
\bibliography{1}
}

\end{document}
