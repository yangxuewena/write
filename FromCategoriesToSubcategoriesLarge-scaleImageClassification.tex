\documentclass[10pt,twocolumn,letterpaper]{article}

\usepackage{cvpr}
\usepackage{times}
\usepackage{epsfig}
\usepackage{graphicx}
\usepackage{amsmath}
\usepackage{amssymb}

\usepackage{times}


\cvprfinalcopy % *** Uncomment this line for the final submission
\def\cvprPaperID{****} % *** Enter the CVPR Paper ID here
\def\httilde{\mbox{\tt\raisebox{-.5ex}{\symbol{126}}}}

\usepackage{indentfirst}
\setlength{\parindent}{2em}
\usepackage{cite}
\usepackage[colorlinks,linkcolor=red,anchorcolor=blue,citecolor=green,backref=page]{hyperref}

\author{Xuewen Yang\\\\
July 14 2018}

\title{From Categories to Subcategories: Large-scale Image Classification with Partial Class Label Refinement}
\begin{document}
\maketitle
\begin{abstract}
The number of digital images is growing extremely
rapidly, and so is the need for their classification. But,
as more images of pre-defined categories become available,
they also become more diverse and cover finer semantic differences.
Ultimately, the categories themselves need to be
divided into subcategories to account for that semantic refinement.
Image classification in general has improved significantly
over the last few years, but it still requires a massive
amount of manually annotated data. 

In this work,the author investigate how coarse category
labels can be used to improve the classification of subcategories.
To this end,the author adopt the framework of Random
Forests and propose a regularized objective function that
takes into account relations between categories and subcategories.
Compared to approaches that disregard the extra
coarse labeled data,they achieve a relative improvement
in subcategory classification accuracy of up to 22\% in our
large-scale image classification experiments.
\end{abstract}
\section{Introduction}
The research community has since then moved to more
challenging, larger datasets, such as ImageNet, which contain thousands of categories and millions of images. In
such datasets, categories are often organized in a hierarchy.
The deeper one goes in the hierarchy, the finer the categories
are and annotated training data becomes rare. In order to obtain
training data for fine subcategories, a natural approach
is to search for images of coarser categories and refine the
labels. This can be very expensive, especially if the subcategories
require expert knowledge\cite{Jia2013Fine} (e.g., breeds of dogs,
bird or flower species, etc.). In this work,the author are interested
in such a scenario where only a subset of the training data
is annotated with fine subcategory labels while the rest has
only coarse category labels (cf . Figure~\ref{fig:1}).
\begin{figure}
\centering
\includegraphics[width=8cm,height=5cm]{1}
\caption{Given training data annotated with a set of categories
like ��feline��, our goal is to refine the classification
into subcategories like ��cat�� and ��lion��. We assume that the
refined labels are available only for a subset of the training
data Sfine, while for the rest, Scoarse, subcategory labels
are not available.}
\label{fig:1}
\end{figure}
\section{Related Work}
Image classification on large-scale data sets has gained
much attention in recent years. While deep convolutional
networks (CNN) achieve high classification accuracy
\cite{Krizhevsky2012ImageNet}, they are computationally intensive and take
weeks to train. Simpler classifiers, such as nearest class
mean classifiers (NCM) combined with a learned metric\cite{Mensink2013Distance}, proved to be a viable alternative with much
shorter running times and near-zero costs for integrating
new classes. In~\cite{Gilliam2014Incremental}, NCMs were integrated in a random
forest framework which increased the performance and
enabled fast incremental learning on a large scale. In this
work,the author address the problem of learning a classifier for
the finest category level when only a part of the training
data is annotated at that level, while the other training samples
have only the labels of a coarser level. This is related
to approaches for semi-supervised or transfer learning.
\section{Regularized NCM Forests}
NCM forests have been shown to provide a good tradeoff between training time and accuracy for large-scale image
classification\cite{Gilliam2014Incremental}.The author first briefly discuss NCM forests.They propose a novel objective
function for training NCM forests, which takes into account
the information gain and the computational cost of a splitting
function to automatically set important parameters of
NCM forests. In this experiments,they show that this modification
improves performance.They then proceed
to show how their classification accuracy can be increased
when only a fraction of the training labels are refined.
\section{Conclusion}
In this paper,the author have addressed the problem of learning
subcategory classifiers when only a fraction of the training
data is labeled with fine labels while the rest only has labels
of coarser categories. To this end,they proposed to use
random forests based on nearest class mean classifiers and
extended the method by introducing a regularized objective
function for training.They also experimentally showed
that the additional training data with the category-only labels
improves the classification of sub-categories up to 22\%
in a large-scale setting. Finally,they have presented experimental
evidence that the approaches taking into account the
hierarchical relations between categories and subcategories
perform better than approaches ignoring these relations.
{\small
\bibliographystyle{ieee}
\bibliography{1}
}

\end{document}
